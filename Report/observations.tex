\section{Observations} % (fold)
\label{sec:observations}

The assignment was simplified by breaking it up into several simple modules.
Many of the modules could be repurposed from previous assignments.
The datapath module was built using an LPM RAM implementation and two modules from Homework 6.
The register file module had to be altered from the one used in Homework 6 to include the output of \verb|RF[0]|.
Since \verb|RegisterFile| was changed to output RQ0, then the \verb|RegisterOEN| module also had to be altered to output whatever contents were in the specified register.
In \verb|RegisterFile| however the only extra \verb|RegisterOEN| output that being used is the output from Register 0.

The program counter was very straightforward.
When compiling the program counter, \emph{Quartus} generates one warning in regards to the clear.
This is because the clear was hardcoded as one of the inputs to the multiplexer, and so it notes that it is stuck as ground.
However, this is intentional and is just an inherent part of the program counter, so the warning was ignored.
The reason the program counter was so simple for this lab was because there were no jump or branch instructions to deal with.
If there had been jump and branch instructions, the program counter would have needed an adder as well as additional logic to be able to change the given address accordingly.

Testing proved very difficult for this assignment.
We tried to simplify the process by testing many of the submodules individually.
This did prove very useful.
However, many processes which were successful on the DE2 board, would not function using \emph{ModelSim}.
At first, many errors were generated by output \verb|reg| types.
Though the DE2 would handle the correctly, \emph{ModelSim} would give them an initial value of \verb|x|.
This caused errors in several of our modules until we recognized the cause.
We were able to avoid this issue by using \verb|initial| statements to create initial values for all of our outputs.
The errors that we still see in \emph{ModelSim} may be from the LPM modules we used which we cannot initialize like our own modules.

We also had difficulties using LPM modules in \emph{ModelSim}.
We were finally able to figure out how to properly import them and where to import them from.
The \emph{Altera} LPM functions are not all held in the same \emph{ModelSim} library.
\emph{ModelSim} needs to import both \verb|220model_ver| and \verb|altera_mf|.

\emph{Quartus} reports tri-state errors in this project because the DE2 does not support internal tri-state wires.
\emph{Quartus} converts these internal tri-stated wires to selectors, and reports this change to us in a warning.
Instead of using the Z state out of the \verb|RegisterOEN| module, we can use a 0 output whenever read enable is not set.
If we do this, however, we then have 16 registers trying to write 0 to each output wire.
One cure for this is to build an OR gate.
The Verilog for this looks a bit unwieldy, as it contains 16 input wires OR-ed together.
This works because we only enable one register output at a time.
We initially implemented this method to avoid the tri-state errors, but \emph{ModelSim} was not able to properly recognize the circuit.

% section observations (end)