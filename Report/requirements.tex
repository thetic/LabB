\section{Requirements} \label{sec:requirements}

\begin{itemize}
    \item Implement the six-instruction programmable processor described in \href{https://moodle.insttech.washington.edu/mod/resource/view.php?id=32929}{Lecture 16b} using Verilog.
    \item Your Quartus project should be in a folder called \verb|LabB| and your top-level module should be called \verb|LabB|.
    Your top-level module will instantiate your processor module and interface to the DE2 board as follows:
    \begin{itemize}
        \item \verb|KEY[0]| acts as your system clock.
        \item \verb|KEY[1]| acts as a synchronous system reset.\footnote{Note that reset will not be able reload memory contents, but it should reinitialize your state machine and reset (clear) \underline{all} your registers.}
        \item Several internal variables are brought to the top-level for debug and display purposes.
        These include the program counter, the instruction register, state machine current state, both inputs to the ALU, the ALU output, the contents of Register File 0, the output of the datapath multiplexer.
        \item \verb|HEX3|, \verb|2|, \verb|1|, and \verb|0| displays the current contents of the IR.
        \item \verb|SW[17:15]| determines what \verb|HEX7|, \verb|6|, \verb|5|, and \verb|4| display as follows:
        \begin{itemize}
            \item 0: \verb|HEX7| = 0; \verb|HEX6|, \verb|HEX5| = PC; \verb|HEX4| = Current State;
            \item 1: \verb|HEX7|, \verb|6|, \verb|5|, \verb|4| = ALU\_A (A-side input to ALU)
            \item 2: \verb|HEX7|, \verb|6|, \verb|5|, \verb|4| = ALU\_B (B-side input to ALU)
            \item 3: \verb|HEX7|, \verb|6|, \verb|5|, \verb|4| = ALU\_Out (ALU output)
            \item 4: Unused (use this for your own debug information)
            \item 5: \verb|HEX7|, \verb|6|, \verb|5|, \verb|4| = Register File 0 contents
            \item 6: \verb|HEX7|, \verb|6|, \verb|5|, \verb|4| = Datapath Multiplexer output
            \item 7: Unused (use this for your own debug information)
        \end{itemize}
    \end{itemize}
    \item A suggested signature for your Processor module:
    \begin{lstlisting}
module Processor( Clk, Reset, IR_Out, PC_Out, StateO, ALU_A, ALU_B, ALU_Out, RQ0, Mux_out);
    input Clk;              // system clock
    input Reset;            // system reset
    output [15:0] IR_Out;   // Instruction register
    output [4:0] PC_Out;    // Program counter
    output [3:0] StateO;    // FSM current state
    output [15:0] ALU_A;    // ALU A-Side Input
    output [15:0] ALU_B;    // ALU B-Side Input
    output [15:0] ALU_Out;  // ALU current output
    output [15:0] RQ0;      // RF[0] contents
    output [15:0] Mux_out;  // Datapath mux output
    \end{lstlisting}
    Feel free to add debug outputs to your processor module.
    \item Turn in the following sample program compiled and loaded into Instruction Memory:
    \begin{lstlisting}
RF[0] = D[A] - D[1A] + D[3] - D[8A];
D[BB] = RF[0];
HALT
    \end{lstlisting}
    Data memory should contain:
    \begin{lstlisting}
D[3] = 0x10AA
D[A] = 0xB0C5
D[1A] = 0x00DC
D[8A] = 0x00E9
    \end{lstlisting}
    \item Make sure that the In System Memory Content Editor can display the contents of both of your memories.
    \item Make sure Quartus recognizes your state machine (as a state machine).
    \item Data memory should be a $256 \times 16$ Quartus RAM LPM with Memory Content Editor enabled.
    \item The Register File should be from Homework 6.
    \item The ALU should be from Homework 6.
    \item The Instruction memory should be a $32 \times  16$ Quartus ROM LPM with Memory Content Editor enabled.
    \item The controller state machine should be similar to the one shown in Lecture 16b.
    \item Implement a ModelSim testbench project that exercises your processor with the required program. You may wish to test with additional programs, but the \verb|*.mif| files you turn in should contain the instructions and data necessary to implement the program shown above.
    \item Write your report using the same outline as before.
    You know by now to be very explicit and detailed about your test procedure and test results.
    Photographs are allowed.
\end{itemize}