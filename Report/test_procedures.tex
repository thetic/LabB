\section{Test Procedures} % (fold)
\label{sec:test_procedures}

The following test procedures will be used to verify that each part of this laboratory exercise satisfies the requirements given in \hyperref[sec:requirements]{Section \ref*{sec:requirements}}.

\subsection{PC Incrementer} % (fold)
\label{sub:pc_incrementer}

The following test procedure will be used to verify that the Verilog module \verb|PCAdder| satisfies the requirements for this part.

%TODO test PCAdder
\begin{enumerate}
    \item The PC incrementer circuit will be tested using \emph{ModelSim} and the testbench shown in %TODO Figure 9, Appendix A.
    This testbench generates the test vectors shown in \hyperref[tab:pcadder_vectors]{Table \ref*{tab:pcadder_vectors}} and outputs the output $O$.
    Simulations will be run in order to verify the behavior shown in \hyperref[tab:pcadder_vectors]{Table \ref*{tab:pcadder_vectors}}.
    \item Generate the test vectors shown in \hyperref[tab:pcadder_vectors]{Table \ref*{tab:pcadder_vectors}} and verify the corresponding outputs, $O$.
\end{enumerate}

\begin{table}[htbp]
    \centering
        \begin{tabular}{ll} \toprule
            $I$     & $O$   \\\midrule
            00000   & 00001 \\
            00001   & 00010 \\
            00010   & 00011 \\
            00011   & 00100 \\
            00100   & 00101 \\
            00101   & 00110 \\
            00110   & 00111 \\
            00111   & 01000 \\
            01000   & 01001 \\
            01001   & 01010 \\
            01010   & 01011 \\
            01011   & 01100 \\
            01100   & 01101 \\
            01101   & 01110 \\
            01110   & 01111 \\
            01111   & 10000 \\
            10000   & 10001 \\
            10001   & 10010 \\
            10010   & 10011 \\
            10011   & 10100 \\
            10100   & 10101 \\
            10101   & 10110 \\
            10110   & 10111 \\
            10111   & 11000 \\
            11000   & 11001 \\
            11001   & 11010 \\
            11010   & 11011 \\
            11011   & 11100 \\
            11100   & 11101 \\
            11101   & 11110 \\
            11110   & 11111 \\
            11111   & 00000 \\\bottomrule
        \end{tabular}
    \caption{PC Incrementer Test Vectors\label{tab:pcadder_vectors}}
\end{table}
% subsection pc_incrementer (end)

\subsection{Multiplexer} % (fold)
\label{sub:multiplexer}

The multiplexer module, \verb|PCUp|, was a simple LPM module and will therefore not be specifically tested.

% subsection multimplexer (end)

\subsection{Decoder} % (fold)
\label{sub:decoder}

The decoder module, \verb|DecoderN|, has been previously used and tested for Homework 6 and will therefore not be specifically tested.
% subsection decoder (end)

\subsection{Register} % (fold)
\label{sub:register}

The register module, \verb|RegisterOEN|, has been previously used and tested for Homework 6 and only minimally modified.
It will therefore not be specifically tested.
% subsection register (end)

\subsection{Controller} % (fold)
\label{sub:controller}

The following test procedure will be used to verify that the Verilog module \verb|controller| satisfies the requirements for this part.

%TODO test controller
\begin{enumerate}
    \item The PC incrementer circuit will be tested using \emph{ModelSim} and the testbench shown in %TODO Figure 9, Appendix A.
    This testbench generates the test vectors shown in \hyperref[tab:controller_vectors]{Table \ref*{tab:controller_vectors}} and outputs listed.
    Simulations will be run in order to verify the behavior shown in \hyperref[tab:controller_vectors]{Table \ref*{tab:controller_vectors}}.
    \item Verify that the RTL Viewer shows a state machine.
    \item Generate the test vectors shown in \hyperref[tab:controller_vectors]{Table \ref*{tab:controller_vectors}} and verify the corresponding outputs.
\end{enumerate}

%TODO write up
\begin{sidewaystable}[htbp]
    \centering\scriptsize
        \begin{tabular}{llllllllllllllll} \toprule
            instruction              & PC\_clr   & PC\_up    & IR\_ld    & D\_addr   & D\_wr & RF\_s & RF\_W\_addr  & RF\_W\_wr    & RF\_Ra\_addr & RF\_Ra\_rd   & RF\_Rb\_addr & RF\_Rb\_rd   & Alu\_s0     & State \\\midrule
            $0000\,0000\,0000\,0000$ &   \\
            $0001\,0000\,0000\,0000$ &   \\
            $0010\,0000\,0000\,0000$ &   \\
            $0011\,0000\,0000\,0000$ &   \\
            $0100\,0000\,0000\,0000$ &   \\
            $0101\,0000\,0000\,0000$ &   \\
        \end{tabular}
    \caption{Controller Test Vectors\label{tab:controller_vectors}}
\end{sidewaystable}
% subsection controller (end)

\subsection{Program Counter} % (fold)
\label{sub:program_counter}

The following test procedure will be used to verify that the Verilog module \verb|PC| satisfies the requirements for this part.

%TODO Test PC
\begin{enumerate}
    \item The program counter circuit will be tested using \emph{ModelSim} and the testbench shown in %TODO Figure 9, Appendix A.
    This testbench generates the test vectors shown in \hyperref[tab:pc_vectors]{Table \ref*{tab:pc_vectors}} and output $O$.
    Simulations will be run in order to verify the behavior shown in \hyperref[tab:pc_vectors]{Table \ref*{tab:pc_vectors}}.
    \item Generate the test vectors shown in \hyperref[tab:pc_vectors]{Table \ref*{tab:pc_vectors}} and verify the corresponding outputs.
\end{enumerate}

\begin{table}[htbp]
    \centering
        %TODO tabular
    \caption{Program Counter Test Vectors\label{tab:pc_vectors}}
\end{table}

% subsection program_counter (end)

\subsection{Instruction Memory} % (fold)
\label{sub:instruction_memory}

The instruction memory module, \verb|imemlpm| is implemented from the LPM.
It will therefore not be specifically tested.
% subsection instruction_memory (end)


\subsection{Instruction Register} % (fold)
\label{sub:instruction_register}

% subsection instruction_register (end)

\subsection{Data Memory} % (fold)
\label{sub:data_memory}

The instruction memory module, \verb|ramlpm| is implemented from the LPM.
It will therefore not be specifically tested.
% subsection data_memory (end)

\subsection{Register File} % (fold)
\label{sub:register_file}

The register file module, \verb|RegisterFile|, has been previously used and tested for Homework 6 and only minimally modified.
It will therefore not be specifically tested.
% subsection register_file (end)

\subsection{Arithmetic Logic Unit} % (fold)
\label{sub:arithmetic_logic_unit}

The arithmetic logic unit module, \verb|ALU|, has been previously used and tested for Homework 6 and will therefore not be specifically tested.
% subsection arithmetic_logic_unit (end)

\subsection{Control Unit} % (fold)
\label{sub:control_unit}

The following test procedure will be used to verify that the Verilog module \verb|cunit| satisfies the requirements for this part.

%TODO test control unit

% subsection control_unit (end)

\subsection{Datapath} % (fold)
\label{sub:datapath}

The following test procedure will be used to verify that the Verilog module \verb|Datapath| satisfies the requirements for this part.

%TODO test datapath

% subsection datapath (end)

\subsection{Processor} % (fold)
\label{sub:processor}

The following test procedure will be used to verify that the Verilog module \verb|Processor| satisfies the requirements for this part.

%TODO test processor

% subsection processor (end)

\subsection{Hex Display} % (fold)
\label{sub:hex_display}

The hex display module, \verb|Hex7seg|, has been previously used and tested for several assignments and will therefore not be specifically tested.
% subsection hex_display (end)

\subsection{Project} % (fold)
\label{sub:project}

The following test procedure will be used to verify that the \emph{Quartus II} project \verb|Lab6| satisfies the requirements for this lab.

%TODO write up project
\begin{enumerate}
    \item The eight instruction processor circuit will be tested using \emph{ModelSim} and the testbench shown in %TODO Figure 9, Appendix A.
    This testbench generates the test vectors shown in \hyperref[tab:project_vectors]{Table \ref*{tab:project_vectors}} and % TODO outputs the multiplexer output M.
    Simulations will be run in order to verify the behavior shown in \hyperref[tab:project_vectors]{Table \ref*{tab:project_vectors}}.
    \item Open the project and verify that compilation produces no errors or unallowed warnings.
    \item Load the project onto the DE2 board without errors.
    \item Generate the test vectors shown in \hyperref[tab:project_vectors]{Table \ref*{tab:project_vectors}} and verify the corresponding outputs,% TODO M.
\end{enumerate}

\begin{table}[htbp]
    \centering
    %TODO tabular
    \caption{Project Test Vectors\label{tab:project_vectors}}
\end{table}
% subsection project (end)

% section test_procedures (end)